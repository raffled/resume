%%%%%%%%%%%%%%%%%%%%%%%%%%%%%%%%%%%%%%%
% Deedy CV/Resume
% XeLaTeX Template
% Version 1.0 (5/5/2014)
%
% This template has been downloaded from:
% http://www.LaTeXTemplates.com
%
% Original author:
% Debarghya Das (http://www.debarghyadas.com)
% With extensive modifications by:
% Vel (vel@latextemplates.com)
%
% License:
% CC BY-NC-SA 3.0 (http://creativecommons.org/licenses/by-nc-sa/3.0/)
%
% Important notes:
% This template needs to be compiled with XeLaTeX.
%
%%%%%%%%%%%%%%%%%%%%%%%%%%%%%%%%%%%%%%

\documentclass[letterpaper]{deedy-resume} % Use US Letter paper, change to a4paper for A4 

\begin{document}

%----------------------------------------------------------------------------------------
%	TITLE SECTION
%----------------------------------------------------------------------------------------

\lastupdated % Print the Last Updated text at the top right

\namesection{Doug}{Raffle}{ % Your name
  \urlstyle{same}\url{http://stat.wvu.edu/~draffle/} \\% Your website, LinkedIn profile or other web address
  \href{mailto:raffled@gmail.com}{raffled@gmail.com} | 724.518.7446 % Your contact information
}

\section{Portfolio}
\begin{flushleft}
This is a collection of projects I've worked on that I think are
particularly cool. Whether created for class projects, formal
research, or just for fun (and practice), I hope you find them as
interesting as I did. Feel free to browse the projects on my
Github page (\urlstyle{same}\url{http://www.github.com/raffled/}), or take the guided tour
of selected projects below. Direct links to the formatted documents
(Shiny apps, knitted HTML, or PDF) are provided, as well as to the
specific repositories containing the source code.\\
\vspace{8pt}

For a more interactive version that's being constantly updated, you
can also check out the portfolio section of my website
(\urlstyle{same}\url{http://stat.wvu.edu/~draffle/portfolio.html})

\vspace{10pt}
\runsubsection{Sentiment Analysis}\\
\descript{Mylan Pharmaceuticals}
While I was toying around with topics to analyze on Twitter, a friend
of mine who works for Mylan pointed me towards the stories that were
flying around about bidding wars between  three generic pharmaceutical 
companies: Mylan, Perrigo, and Teva.  Mylan is a pillar of the
Morgantown community, so their being bought out would be huge news in
Morgantown.\\
\vspace{10pt}

We pulled the most recent tweets with the hashtag \#Mylan on April 24,
2015 and performed sentiment analysis to see what the Twittersphere
had to say about the generics wars.\\
\vspace{10pt}

\end{flushleft}
\begin{tabular}{l  l}
  Analysis: & \urlstyle{same}\url{http://stat.wvu.edu/~draffle/portfolio/mylanSentiment.html}\\
  Source Code: & \urlstyle{same}\url{http://github.com/raffled/Sentiment/}\\
\end{tabular}


\vspace{10pt}
\begin{flushleft}
\runsubsection{USGS Seismic Data}\\
The USGS data set includes information about 20,000 global seismic events
that occured from January 1, 2015 to April 15, 2015.\\
\vspace{10pt}

\descript{Interactive Map}
First, I created an interactive map showing the location of the seismic events using 
\urlstyle{same}\href{http://www.rstudio.com}{RStudio's}
\urlstyle{same}\href{http://shiny.rstudio.com/}{Shiny}package.  Each
point represents one event.  Events are colored according to event
type (earthquake, exposion, etc.) and scaled by magnitude. Because
plotting all the events takes a while, the map is restricted to the
8,480 events which occured in the six weeks of March 1, 2015 - April
15, 2015. \\
\vspace{10pt}

\descript{Exploratory Data Analysis}
This EDA is a basic exploration of the USGS data set, written in R
Markdown, including numerical and graphical summaries of the data
set.\\
\vspace{10pt}

\descript{Classification}
First, I wanted to play around with some classification models, so I
wanted to see if the three variable describing the seismic event could
be used to predict the type of event. While not a particularly 
practical problem (if there's a large seismic even near me, I care
more about the magnitude than what caused it), it is particularly
challenging because of the nature of the data: 96.4\% of the
observations are earthquakes.  I tuned several different models and
compared their results.\\
\vspace{10pt}

\end{flushleft}
\begin{tabular}{l  l}
  Map: & \urlstyle{same}\url{http://raffle.shinyapps.io/seismic}\\
  EDA: & \urlstyle{same}\url{http://stat.wvu.edu/~draffle/portfolio/seismicEDA.html}\\
  Classification: & \urlstyle{same}\url{http://stat.wvu.edu/~draffle/portfolio/seismicModelling.html}\\
  Source Code: & \urlstyle{same}\url{http://github.com/raffled/seismic/}\\
\end{tabular}



\end{document}